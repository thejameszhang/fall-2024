\documentclass[12pt]{scrartcl}
\usepackage[sexy]{james}
\usepackage[noend]{algpseudocode}
\setlength {\marginparwidth}{2cm}
\usepackage{answers}
\usepackage{array}
\usepackage{tikz}
\newenvironment{allintypewriter}{\ttfamily}{\par}
\usepackage{listings}
\usepackage{xcolor}
\usetikzlibrary{arrows.meta}
\usepackage{color}
\usepackage{mathtools}
\newcommand{\U}{\mathcal{U}}
\newcommand{\E}{\mathbb{E}}
\usetikzlibrary{arrows}
\Newassociation{hint}{hintitem}{all-hints}
\renewcommand{\solutionextension}{out}
\renewenvironment{hintitem}[1]{\item[\bfseries #1.]}{}
\renewcommand{\O}{\mathcal{O}}
\declaretheorem[style=thmbluebox,name={Chinese Remainder Theorem}]{CRT}
\renewcommand{\theCRT}{\Alph{CRT}}
\setlength\parindent{0pt}
\usepackage{sansmath}
\usepackage{pgfplots}

\usetikzlibrary{automata}
\usetikzlibrary{positioning}  %                 ...positioning nodes
\usetikzlibrary{arrows}       %                 ...customizing arrows
\newcommand{\eqdef}{=\vcentcolon}
\newcommand{\tr}{{\rm tr\ }}
\newcommand{\im}{{\rm Im\ }}
\newcommand{\spann}{{\rm span\ }}
\newcommand{\Col}{{\rm Col\ }}
\newcommand{\Row}{{\rm Row\ }}
\newcommand{\dint}{\displaystyle\int}
\newcommand{\dt}{\ {\rm d }t}
\newcommand{\PP}{\mathbb{P}}
\newcommand{\horizontal}{\par\noindent\rule{\textwidth}{0.4pt}}
\usepackage[top=3cm,left=3cm,right=3cm,bottom=3cm]{geometry}
\newcommand{\mref}[3][red]{\hypersetup{linkcolor=#1}\cref{#2}{#3}\hypersetup{linkcolor=blue}}%<<<changed

\tikzset{node distance=4.5cm, % Minimum distance between two nodes. Change if necessary.
         every state/.style={ % Sets the properties for each state
           semithick,
           fill=cyan!40},
         initial text={},     % No label on start arrow
         double distance=4pt, % Adjust appearance of accept states
         every edge/.style={  % Sets the properties for each transition
         draw,
           ->,>=stealth',     % Makes edges directed with bold arrowheads
           auto,
           semithick}}


% Start of document.
\newcommand{\sep}{\hspace*{.5em}}

\pgfplotsset{compat=1.18}
\begin{document}
\title{MATH403: Abstract Algebra}
\author{James Zhang\thanks{Email: \mailto{jzhang72@terpmail.umd.edu}}}
\date{\today}

\definecolor{dkgreen}{rgb}{0,0.6,0}
\definecolor{gray}{rgb}{0.5,0.5,0.5}
\definecolor{mauve}{rgb}{0.58,0,0.82}

\lstset{frame=tb,
  language=Java,
  aboveskip=3mm,
  belowskip=3mm,
  showstringspaces=false,
  columns=flexible,
  basicstyle={\small\ttfamily},
  numbers=left,
  numberstyle=\tiny\color{gray},
  keywordstyle=\color{blue},
  commentstyle=\color{dkgreen},
  stringstyle=\color{mauve},
  breaklines=true,
  breakatwhitespace=true,
  tabsize=3
}


\maketitle
These are my notes for UMD's MATH403: Abstract Algebra. These notes are taken live in class 
(``live-\TeX``-ed). This course is taught by Professor Qendrim Gashi, qgashi@umd.edu. The textbook for the class is 
\textit{Contemporary Abstract Algebra} by Joseph A. Gallian. \textit{$10$th Edition}.
\tableofcontents

\newpage

\section{Preliminaries}

\begin{definition}[Well-Ordering Principle]
  If $\emptyset \neq S \subseteq \NN \implies S$ has a smallest element, 
\end{definition}

\begin{theorem}[Division Algorithm]
  Suppose $a, b \in \ZZ, \text{ s.t. } a < b \implies \exists! \  q, r \in \ZZ, 0 \leq r < b$ 
  such that $a = bq + r$.

  \begin{proof}

    \hfill

    Define a set $\{a - bk \ | \ k \in \ZZ, a - bk \geq 0\} \neq \emptyset$. By the Well-Ordering Principle, 
    it has a smallest element which we will denote by $r$, and $r = a - bq \ \forall \ q \in \ZZ$.

    Assume $r \geq b$. Therefore, 
    \[0 \leq r - b = a * bq - b = a - b(q+1) \in S\]
    which is a contradiction since $r$ is the smallest element. If there exists $q!, r!$ of the same 
    type of $q, r$ then 
    \[bq + r = bq! + r! \implies b(q - q!) = r! - r\]
    We can assume that $r! > r$, so $b \ | \ r! - r$, therefore $r! = r!$ and $q! = q!$.
  \end{proof}
\end{theorem}

\begin{lemma}[Bezout's Lemma]
  Let $a, b \in \ZZ \backslash \{0\}$ then $\exists \ s, t \in \ZZ$ such that $GCD(a, b) = as + bt$
  and $GCD(a, b)$ is the least (positive) integer expressed in such a linear combination.

  \begin{proof}
    
    \hfill

    Define the set $S = \{am + bn \ | \ m, n \in \ZZ, am + bn > 0\} \neq \emptyset$. By the 
    Well-Ordering Principle, let $d = \min S$, which by definition of the set, must have a form 
    $d = as + bt$ for some $s, t \in \ZZ$. Now we have to prove that $d$ is a divisior of $a, b$ and that 
    it is the greatest divisior.

    Claim $1$: $d$ is a divisor of $a, b$. By the Division Algorithm, $a = qd + r$ for some $q, r \in \ZZ, 0 \leq r < d$.
    If $r > 0$, then
    \[r = a - qd = a - q(as + bt) = a(1-qs) + b(-qt) \in S\]
    which is a contradiction because $r \in S$ but $r < d$ and $d$ is the smallest element in the set, so $r$ cannot be in the set. 
    
    Claim $2$: Any common divisor of $a$ and $b$ divides $d$. Assume $d$ is such a divisor. Therefore, 
    we write $a = d' h, b = d'k$. Therefore, 
    \[d = as + bt = s(d'h) + t(d'k) = d'(hs + kt)\]
    and so we get that $d' \leq d$ and so $d = GCD(a, b)$. 

  \end{proof}
\end{lemma}

\begin{corollary}
  If $a, b$ are relatively prime $\Longleftrightarrow \exists \ s, t \in \ZZ \text{ s.t. } as + bt = 1$
\end{corollary}

\begin{note}
  Define the GCD operator $GCD(a, b) = (a, b)$ for two integers $a, b \in \ZZ$.
\end{note}

\begin{example}
  Let $n \in \NN$ and consider $(n^2 + n + 1, n + 1)$. Note that 
  \[(n^2 + n + 1) + (n + 1)(-n) = 1 \overset{\text{Corollary}}{\Longrightarrow} (n^2 + n + 1, n + 1) = 1\]
\end{example}

\begin{theorem}[Fundamental Theorem of Arithmetic]
  Given $n \in \NN, \ \exists! \ p_i$ primes and $t_i \in \NN, i, \ldots, k$ such that 
  \[n = \prod_{i=1}^k p_i^{t_i}\]
  where $p_i \neq p_j$ and uniqueness is up to reordering.

  \begin{proof}
    Base case: for $n = 2$ this is true. Inductive hypothesis: assume this is true for $n \leq m$. 
    Inductive case: now consider $n = m +1$. If this number is prime, then we are done. If it is not a prime, 
    then by definition of not prime, then $m + 1 = m_1 \cdot m_n, m_1, m_2 \in \NN$ such that 
    $1 < m_1, m_2 < m + 1$. By assumption in our inductive hypothesis,
    \begin{align*}
      m_1 = \prod_{i=1}^{k_1} p_i^{t_i} && m_2 = \prod_{i=1}^{k_2} (p_i')^{k_i'} \implies m_1m_2 = \left(\prod_{i=1}^{k_1} p_i^{t_i}\right)\left(\prod_{i=1}^{k_2} (p_i')^{k_i'}\right)
    \end{align*}
    and thus we have proved existence. To prove uniqueness, we will use Euclid's Lemma

    \begin{lemma}[Euclid's Lemma]
      If you $p$ prime, $a, b \in \NN$ then $p | (ab) \implies p|a \lor p|b$

      \begin{proof}
        
      \end{proof}
    \end{lemma}

    Suppose there are two expressions where $p_i, q_j$ are prime and distinct 
    \[n = p_1^{t_1}p_2^{t_2} \cdots p_k^{t_k} = q_1^{s_1}q_2^{s_2}\cdots q_c^{s_c}\]
    Since $p_1 \ | \ (q_1^{s_1}, \ldots, q_c^{s_c}) \implies \ \exists \ j, p_i \ | \ q_j \implies p_i = q_j$. 
    By repeating this process, $k=c$, and $p_i$'s are just reordering of $q_j$'s and the 
    powers agree.
  \end{proof}
\end{theorem}

\subsection{Relations}

\begin{definition}[Relation]
  Suppose $A, B \neq \emptyset$, define $R \subset A \times B$, so $R$ is a \vocab{relation} between
  $A$ and $B$. If $A = B$, they say $R$ is a relation on $A \implies R \subset A \times A$. 
\end{definition}

\begin{note}[Functions are Relations]
  Suppose $f: A \times B$, then $f \subset A \times B = \{(fa, f(a)): a \in A\}$. Similarly, for a $m \times n$ matrix 
  that is a linear transformation, then $f: [m] \times [n] \to \RR, m = \{1, 2, \ldots, m\}$ and $n = \{1, 2, \ldots, n\}$.
\end{note}

\begin{definition}[Equivalence Relations]
  Suppose $R \subset A \times A$ then it must satisfy three properties
  \begin{enumerate}
    \item $R$ is reflexive, which means that $(a,a) \in \RR \ \forall \ a \in A$.
    \item $R$ is symmetric, which means $(a,b) \in \RR \implies (b, a) \in \RR \ \forall \ a, b \in \RR$
    \item $R$ is transitive, which means $(a, b) \in R \ \land \ (a, c) \in R \implies (a,c) \in R$
  \end{enumerate}
\end{definition}

\begin{note}
  Instead of writing $(a, b) \in R$, let us write $aRb$. 
\end{note}

\begin{example}
  Let $R \subset \RR \times \RR$, and so $a=a \ \forall \ a \in \RR$ and $a = b \implies b = a \ \forall \ a, b \in \RR$ and 
  $(a = b \ \land \ b = c) \implies a = c \ \forall \ a, b \in \RR$ 
\end{example}

\begin{definition}[Equivalence Class]
  Let $R \subset A \times A$ be an equivalence relation, then for any $x \in A$ we call 
  $C_x := \{a \in A \ | \ xRa\}$ 
\end{definition}
  
\begin{note}
  Note that $x \in C_x$ by reflexivity and so $C_x \neq \emptyset$
\end{note}

\begin{note}
  If $x, y \in A$ then $C_x = C_y$ or $C_x \cap C_y = \emptyset$ and this naturally leads to the 
  ideas of partitions, since $A = \cup_{x \in A} C_x$ where all of the $C_x$'s are disjoint of equal.
\end{note}

\begin{definition}[Modular Arithmetic]
  Fix $m \in N$, usually $ m > 1$. Consider residues when dividing any integer by $m = 3$. 
  \[\begin{pmatrix}
    6 & 7 & 8\\
    3 & 4 & 5\\
    0 & 1 & 2\\
    -3 & -2 & -1\\
    \vdots & \vdots & \vdots
  \end{pmatrix}\]
  where the columns of this matrix are equivalence classes $C_0, C_1, C_2$. Denote 
  $\{\overline{0}, \overline{1}, \ldots, \overline{n-1}\} = \{0, 1, \ldots, n-1\}$ where each of these are the equivalence classes corresponding to these 
  integers. In this way (lol), $1 + 2 = 0$ where addition does not depend on the equivalence class; this just means we don't have 
  to stay in the same row when performing the addition.
\end{definition}

\begin{note}
  Let $[n] = \{1, 2, \ldots, n\}$ and denote by $S_n$ the set of all bijective maps $[n] \to [n]$. If $f, g \in S_n$
  then $f \circ g \in S_n$, and so $(S_n, \circ)$ is known as the \vocab{permutation group}. This composition of maps is the first algebraic operation that gives way to groups.
\end{note}


\section{Groups}

\end{document}

