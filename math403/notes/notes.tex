\documentclass[12pt]{scrartcl}
\usepackage[sexy]{james}
\usepackage[noend]{algpseudocode}
\usepackage{answers}
\usepackage{array}
\usepackage{tikz}
\setlength {\marginparwidth}{2cm}
\newenvironment{allintypewriter}{\ttfamily}{\par}
\usepackage{listings}
\usepackage{xcolor}
\usetikzlibrary{arrows.meta}
\usepackage{color}
\usepackage{mathtools}
\newcommand{\U}{\mathcal{U}}
\newcommand{\E}{\mathbb{E}}
\usetikzlibrary{arrows}
\Newassociation{hint}{hintitem}{all-hints}
\renewcommand{\solutionextension}{out}
\renewenvironment{hintitem}[1]{\item[\bfseries #1.]}{}
\renewcommand{\O}{\mathcal{O}}
\declaretheorem[style=thmbluebox,name={Chinese Remainder Theorem}]{CRT}
\renewcommand{\theCRT}{\Alph{CRT}}
\setlength\parindent{0pt}
\usepackage{sansmath}
\usepackage{pgfplots}

\usetikzlibrary{automata}
\usetikzlibrary{positioning}  %                 ...positioning nodes
\usetikzlibrary{arrows}       %                 ...customizing arrows
\newcommand{\eqdef}{=\vcentcolon}
\usepackage[top=3cm,left=3cm,right=3cm,bottom=3cm]{geometry}
\newcommand{\mref}[3][red]{\hypersetup{linkcolor=#1}\cref{#2}{#3}\hypersetup{linkcolor=blue}}%<<<changed

\tikzset{node distance=4.5cm, % Minimum distance between two nodes. Change if necessary.
         every state/.style={ % Sets the properties for each state
           semithick,
           fill=cyan!40},
         initial text={},     % No label on start arrow
         double distance=4pt, % Adjust appearance of accept states
         every edge/.style={  % Sets the properties for each transition
         draw,
           ->,>=stealth',     % Makes edges directed with bold arrowheads
           auto,
           semithick}}


% Start of document.
\newcommand{\sep}{\hspace*{.5em}}
\pgfplotsset{compat=1.18}
\begin{document}
\title{MATH403: Abstract Algebra}
\author{James Zhang\thanks{Email: \mailto{jzhang72@terpmail.umd.edu}}}
\date{\today}

\definecolor{dkgreen}{rgb}{0,0.6,0}
\definecolor{gray}{rgb}{0.5,0.5,0.5}
\definecolor{mauve}{rgb}{0.58,0,0.82}

\lstset{frame=tb,
  language=Java,
  aboveskip=3mm,
  belowskip=3mm,
  showstringspaces=false,
  columns=flexible,
  basicstyle={\small\ttfamily},
  numbers=left,
  numberstyle=\tiny\color{gray},
  keywordstyle=\color{blue},
  commentstyle=\color{dkgreen},
  stringstyle=\color{mauve},
  breaklines=true,
  breakatwhitespace=true,
  tabsize=3
}

\maketitle
These are my notes for UMD's MATH 403: \textit{Abstract Algebra}. They are taken live during class. This course is taught by Dr. Jonathan Rosenberg.
\tableofcontents
\newpage

\section{August 26, 2024}
We start today by reviewing the syllabus, what this course covers, the textbook (\textit{Contemporary Abstract Algebra} by Jonathan Gallian).

\subsection{Structures in Abstract Algebra}
There are two main algebraic objects: \vocab{groups} and \vocab{rings}. Within groups are \vocab{semigroups}; one more axiom gives a \vocab{monoid}, one more on top of that gives \vocab{groups}. We will primarily discuss groups.

\begin{definition}
    A \vocab{semigroup} is a set $S$ with a multiplication operation $x : S \times S \to S$, or $(x, y) \mapsto x \times y$.
\end{definition}

\begin{remark}
    The associative law holds for the semigroup multiplication operator, e.g. $(x \times y) \times z = x \times (y \times z)$. Also, the $xyz$ for the triple product is unambiguous.
\end{remark}

\begin{definition}
    A \vocab{monoid} is a semigroup with a special element $1$ (sometimes denoted by $0$, or $e$) such that for every $x \in S$, $1 \cdot x = x \cdot 1 = x$.
\end{definition}

\begin{definition}
    A \vocab{group} is a monoid with an inversion operator $x \mapsto x^{-1}$ such that for every $x \in S$, $x \cdot x^{-1} = x^{-1} \cdot x = 1$.
\end{definition}

\begin{remark}
    The inversion operator in groups makes it possible to do cancellation, so if $x \times y = x \times z$, then $x^{-1}(xy) = x^{-1}(xz)$ leads to $(x^{-1}x)y = (x^{-1}x)z \to y = z$.
\end{remark}

Groups arise in practice from \vocab{symmetries}. The usual symbol of the ordinary integers if $\ZZ$. A subset of the integers are the natural numbers $\NN$.

\begin{remark}
    $\ZZ$ has an \vocab{order}, i.e. for any $a, b \in \ZZ$, one of the following holds:
        \begin{align*}
            \begin{cases}
                a < b \\
                a > b \\
                a = b
            \end{cases}
        \end{align*}
    $\ZZ$ also has the \vocab{well-ordering property}: and nonempty subset of $\NN$ has a unique smallest element. This is what makes it possible to do mathematical induction.
\end{remark}

\begin{definition}
    A \vocab{ring} is an algebraic system with two associative operations: addition and multiplication.
\end{definition}

\begin{remark}
    The distributive law holds for rings, e.g. $a(b+c) = ab + ac$. 
\end{remark}

\begin{proposition}
    Suppose $a, b \in \ZZ$ with $b > 0$. Then there is a unique way to divide $a$ by $b$ and get a remainder $r$, e.g. $a = qb + r$, where $ 0 \leq r \leq b-1$.
\end{proposition}

\begin{definition}
    We say $a$ \vocab{divides} $b$ if $r = 0$.
\end{definition}

\section{August 28, 2024}
Recall last class, where we introduced Proposition 1.9. We will now introduce the \vocab{Division algorithm}:

\begin{theorem}[Division algorithm]
    If $a, b \in \ZZ$, where $b > 0$, one can write $a = qb + r$, where $q, r$ are unique integers and $0 \leq r < b$. Here, $r$ is called the \vocab{remainder} and $q$ is called the \vocab{quotient}.
\end{theorem}

\begin{definition}
    We say (for $b \neq 0$) that $b|a$ (said $b$ divides $a$) if $a = qb$ for some $q \in \ZZ$.
\end{definition}

\begin{remark}
    If $a > 0$, all (positive) divisors of $a$ are $\geq 1$, $\leq a$.
\end{remark}

\begin{definition}
    We say that $a > 0$ is \vocab{prime} if $a \neq 1$ and its only positive divisors are $1$ and $a$.
\end{definition}

\begin{theorem}
    Let $a, b \in \ZZ - \{a\}$. Then,
        \begin{enumerate}
            \item There exists a unique largest positive number that divides both $a$ and $b$. This is called the \vocab{greatest common divisor} of $a$ and $b$, denoted by gcd($a$, $b$).

            \item gcd$(a, b) = \min\{t = ma + nb > 0 \ | \ m, n \in \ZZ\}$

            \item Any common divisor of $a$ and $b$ divides $\text{gcd}(a, b)$.
        \end{enumerate}
\end{theorem}

\begin{proof}
    We will prove each part separately:
        \begin{enumerate}
            \item Note that the set of positive common divisors is nonempty (e.g. it contains $1$) but lies in $\{1, 2, \ldots, |a|\}$. So this set is finite and has a maximum, proving the existence of the greatest common divisor.

            \item We will prove 2. and 3. together. Let $S = \{t = ma + nb > 0 \ | \ m, n \in \ZZ\}$; note that $S$ is a subset of $\NN^+$. By the well-ordering principle, $S$ has a unique smallest element $t_0$. By the division algorithm, $a = qb + r$ with $0 \leq r < t_0$. Since $t_0 = ma + nb$ for some $m, n$, we see that $r = a-qt_0 = a-qma-qnb = (1-qm)a + (-qn)b$. Thus, either $r = 0$ or else $r \in S$; since $r < t_0$, we have $r \notin S$. By the same argument, $t_0 | b$, so $t_0$ is a common divisor of $a$ and $b$. Let $d$ be any common divisor of $a, b$. So, $a = du$, $b = dv$ for some $u, v \in \ZZ$. But $t_0 = ma + nb$ for some $m, n$, so $t_0 = mdu + ndr = d(mu + nv)$ is a multiple of $d$, i.e. $d|t_0$. So if $ d \leq t_0$, we have $t_0 = \text{gcd}(a, d)$ and any common divisor of $a, b$ divides $\text{gcd}(a, b)$, as desired.
        \end{enumerate}
\end{proof}

\begin{corollary}
    If $a, b$ have no common divisors except $\pm 1$, then $1 = ma + nb$ for some $m, n \in \ZZ$.
\end{corollary}

There is an algorithm for finding teh $m, n$, called the \vocab{Euclidean algorithm}. We present an example below:

\begin{example}
    Find $m$, $n$ for $a = 13$ $b = 54$.
\end{example}
\begin{soln}
    By the division algorithm, we have the following:
            \begin{align*}
                54 &= 4 \cdot 13 + 2 \\
                13 &= 6 \cdot 2 + 1 \\
            \end{align*}
        Working backwards, we obtain the following:
            \begin{align*}
                1 &= 13 - 6 \cdot 2 \\
                  &= 13 - 6 \cdot (54 - 4 \cdot 13) \\
                  &= 25 \cdot 13 - 6 \cdot 54
            \end{align*}
        Thus, $m = 25$, $n = -6$.
\end{soln}

\begin{theorem}
    If $p$ is prime and $p|ab$, then $p|a$ or $p|b$ or both.
\end{theorem}

\begin{proof}
    If $p|a$, we're done. Assume $p \nmid a$. Let $t = \text{gcd}(p, b)$. Since $p$ is prime, $t = 1$ or $p$. If it's $p$, we're done. But if $t = 1 = mp + nb$, then $a = mpa + nba$. But $p|ab$, so $p$ divides both terms on the right, and thus $p|a$, as desired.
\end{proof}

\begin{theorem}[Fundamental Theorem of Arithmetic]
    Any positive integer can be written as a product of primes. The decomposition is unique except for the order of the factors.
\end{theorem}

\begin{proof}
    Let $a > 0$. The set of positive divisors of $a$ is either $\{1\}$, in which case $a = 1$, or else $S = \{d > 1 \ | \| d|a\}$ is nonempty. In this case, $S$ has a minimum by the well-ordering principle. This must be a prime; factor it and repeat, which gives a factorization into primes. To prove uniqueness, suppose $p_1 \cdots p_r = q_1 \cdots q_s$ with $p_j$, $q_k$ primes (repetitions allowed). So $p_1|q_1 \cdots q_s$, so by the a previous theorem, $p_1$ divides some $q_k$, hence $p_1 = q_k$. After re-indexing, we obtain $p_1 \cdots p_r = p_1q_1 \cdots q_s$. Cancel $p_1$ and repeat, finishing the proof.
\end{proof}

\begin{remark}
    The convention we will use is that the produce of the empty set of primes is $1$. A slogan we will use if $\ZZ$ is a \vocab{unique factorization domain}.
\end{remark}

\section{August 30, 2024}
Today, we will start discussing \vocab{sets}.

\subsection{Sets}
A set $S$ has objects $x$; we write $x$ is an element of $S$ as $x \in S$> A subset $A \subset S$ is a subcollection of $S$. Any set $S$ has a \vocab{cardinality} (``size"), often denoted by $|S|$. A finite set $S$ has cardinality in $\NN = \{0, 1, 2, 3, \ldots\}$. An infinite set can be characterized by a number of properties to follow.

\begin{remark}
    Not all integer sets have the same cardinality, but omost in this course will have cardinality $\mathcal{X}_0 = |\ZZ| = |\NN|$.
\end{remark}

\begin{remark}
    Another way of denoting cardinality of a set is as follows: if $\mathcal{X}_0$ is a set, 
\end{remark}

\begin{fact}
    Saying a set is infinite is the same way of saying that it has the same cardinality as some proper subset.
\end{fact}

\begin{definition}
    A proper subset $A$ of a set $S$ is such that $A \subset S$ and $A \neq S$.
\end{definition}

\begin{definition}
    A function has a \vocab{graph} which is a subset of $X \times Y$. Note that not every subset of $X \times Y$ is the graph of a function.
\end{definition}

\begin{definition}
    If $S \subseteq X \times Y = \{(x, y) : x \in X, y \in Y\}$, $S$ is the graph of a function $X \to Y$. This is the same as saying for every $x \in X$ there is a unique $y \in Y$ with $(x, y) \in s$
\end{definition}

\begin{definition}
    Given a function $f : X \to Y$, $X$ is called the \vocab{domain} of the function and $Y$ is called the \vocab{codomain} of the function. The \vocab{range} of $f$ is the set of all $\{f(x) : x \in X\} \subseteq Y$.
\end{definition}

\begin{definition}
    A function is called \vocab{one-to-one} or \vocab{injective} if whenever $x_1 \neq x_2$, $f(x_1) \neq f(x_2) \in Y$. A function is called \vocab{onto} or \vocab{surjective} if for every $y \in Y$, there is \textit{some} $x \in X$ with $f(x) = y$. A function is called \vocab{bijective} if it is both injective and surjective. This is equivalent to $f$ being invertible, i.e. to there existing $g : Y \to X$ such that $g \circ f = \text{id}_X$ and $f \circ g = \text{id}_Y$, where $g \circ f = g(f(x))$.
\end{definition}

\begin{remark}
    Cantor defined cardinality by saying that $|S_1| = |S_2|$ if there is a bijective function $f : S_1 \to S_2$. He also defined a set $S$ to be \vocab{infinite} if there is an injective function $f : S \to S$ which is not surjective.
\end{remark}

We now present an example of the above.

\begin{example}
    Consider $S = \NN = \{0, 1, 2, \ldots\}$. Take $f(x) = x+1$; this has range $\{1, 2, 3, \ldots\} \not\subset \NN$.
\end{example}

For finite sets, the situation is different. On a  finite set $S$, every injective function $S \to S$ is surjective.

\begin{fact}
    If $S$ is finite and $f : S \to S$ is surjective, then $f$ is injective. Cantor also defined $|S_1| \leq |S_2|$ if there is an injective function $S_1 \to S_2$. 
\end{fact}

A non-obvious fact is that the above is equivalent to the existence of a surjective function $S_2 \to S_1$.

\begin{theorem}[Pigeonhole Principle]
    Suppose $S_1$ and $S_2$ are finite sets with $|S_1| < |S_2|$. Then for any function $f: S_2 \to S_1$, there exists $x \in S_1$ with $|\{y \in S_2 : f(y) = x\}| \geq 2|$, where $\{y \in S_2 : f(y) = x\} = f^{-1}(x)$. This is called the \vocab{Pigeonhole Principle}.
\end{theorem}

\subsection{Equivalence relations}

Often we will have a set $S$ and want to \vocab{partition} it into pieces. A partition is a family of disjoint subsets of $S$ whose union is all of $S$. A partition is equivalent to defining $R \subset S \times S$ (where $R$ is the \vocab{relation}) with the following properties:
    \begin{enumerate}
        \item $\forall x \in S, (x, x) \in \RR$
        \item $\forall x, y \in S, (x, y) \in \RR \iff (y, x) \in \RR$
        \item $\forall x, y, z \in S$, $(x, y) \in \RR$ and $(y, z) \in \RR \implies (x, z) \in \RR$
    \end{enumerate}

The above three properties are called the \vocab{reflexive}, \vocab{symmetric}, and \vocab{transitive} properties; the relationship is called an \vocab{equivalence relation}. We will continue with an example of an equivalence relation next lecture.

\section{September 4, 2024}

\begin{example}
    Fix $n > 0$ in $\NN$ and let $X = \ZZ$. Say that $x \sim y$ if $x-y$ is divisible by $n$ (this is often written as $x \equiv y \pmod n$). This is an equivalence relation since
        \begin{enumerate}
            \item $n|(x-x=0) \ \forall x$
            \item If $n|(x-y)$, then $n|(y-x)$
            \item If $n|(x-y)$ and $n|(y-z)$, then $x-z = (x-y) + (y-z)$, which is divisible by $n$
        \end{enumerate}
\end{example}

The equivalence classes are denoted $\ZZ / n$. These equivalence classes can be labeled by $0, 1, \ldots, n-1$ because of the division algorithm: $\forall x \in \ZZ$, $x = nq + r$ with $ \in \{0, 1, \ldots, n-1\}$ and this decomposition is unique.

\begin{fact}
    The usual operations $+$, $\times$ on $\ZZ$ pass to equivalence classes, i.e. if $x \sim x'$, $y \sim y'$, $z \sim z'$, $x+y \sim x' + y'$, $x \times y \sim x' \times y'$. Why? Note that
        \begin{align*}
            x \times y - x' \times y' &= x \times y - x \times y' + x \times y' - x' \times y' \\
            &= x \times (y-y') + (x-x') \cdot y'
        \end{align*}
    As $y-y'$ and $x-x'$ are both multiples, of $n$, we have that $x \times (y-y') + (x-x') \cdot y'$ is also a multiple of $n$.
\end{fact}

So $\ZZ / n$ with addition is a group, denoted by Gallian as $Z_n$ (Rosenberg prefers $\ZZ_n$). Note that $(\ZZ / n, x)$ is not a group, since $0$ has no multiplicative inverse. But the invertible elements of $\ZZ / n$ form a group $(\ZZ / n)^x$, which Gallian denotes $U(n)$. \\

An equivalence relation $R$ on a set $X$ defines a partition of $X$ into \vocab{equivalence classes} $\{x \in X : xRy\}$ for some fixed $y$. \\

The operations $+, \times$ on $\ZZ / n$ define \vocab{modular arithmetic} $\pmod n$. These satisfy all the usual rules of arithmetic (commutative, associate, distributive).

\begin{example}
    Prove that if $n$ is a positive integer, $n^3 + (n+1)^3 + (n+2)^3$ is a multiple of $9$.
\end{example}

\begin{proof}
    We work $\pmod 9$. Note that this repeats in cycles of $3$, e.g. for $n \equiv \{0, 1, 2, 3, \ldots\} \pmod 9$, $n^3 \equiv \{0, 1, -1, 0, \ldots\} \pmod 9$. In all cases, the sum of $3$ consecutive numbers is $0$.
\end{proof}

\begin{example}[Symmetries of a polygon in $\RR^2$]
    Consider a regular polygon in $\RR^2 = \CC$ with $n$ sides. If you don't like complex numbers, take the polygon $P$ with vertices $\left(\cos \frac{2\pi k}{n}, \sin\frac{2\pi k}{n}\right)$, where $k = 0, 1, \ldots, n-1$. What are the \vocab{symmetries} of $P$? A \vocab{symmetry} means a distance-preserving map $\RR^2 \to \RR^2$ sending $P$ to itself. The set of such is called $D_n$.
\end{example}

\begin{fact}
    Each element of $D_N$ is a rotation $R_{\theta}$ by an angle $\theta$ (a multiple of $\frac{2\pi}{n}$) or a \vocab{reflection} across an axis. The set of rotations is $\left\{R_0 = I, R_{\frac{2\pi}{n}}, R_{\frac{4\pi}{n}}, \ldots, R_{\frac{(n-1)\pi}{n}}\right\}$. There are exactly $n$ of these.
\end{fact}

How many reflections are there in $D_n$? $n$. If $N$ is even we have the axes passing through $2$ opposite vertices and axes passing midway through a pair of opposite sides. If $n$ is odd, each reflection must have a fixed point. Again have $n$ axes of reflection and $N$ reflection operations. So $|D_n| = n + n = 2n$.

\section{September 6, 2024}
We start by going over an example of a group from last lecture.

\begin{example}
    Suppose we have a $P$ regular polynomial with $n$ sides $\{0, 1, \ldots, n-1\}$. The \vocab{dihedral group} $D_N$ preserves distances. So if two vertices are adjacent, they stay adjacent.
\end{example}


\subsection{Semigroups}
\begin{definition}
    A \vocab{semigroup} $S$ is a set with a multiplication operation $m : S \times S \to S$ satisfying the associative rule $m(m(a, b), c) = m(a, m(b, c))$. Usually we suppress the $M$ and just write $(ab)c = a(bc)$.
\end{definition}

Semigroups have very few good properties: usually cancellation fails, i.e. $ab = ab \not\to b = c$ and $ba = bc \not \to b = c$. If you fix an element $e$, $m(a, b) = e$ for all $a, b$ satisfies the associative rule.

\subsection{Monoids}

\begin{definition}
    Better than semigroups is what is called a \vocab{semigroup with identity} or a \vocab{monoid}. We add the axiom that there is an element $e$ such that $ae = ea = a$ for all $a \in S$.
\end{definition}

\begin{remark}
    $e$ above is unique with this property, since if $e'$ has the same property then $e=e'e = e'$.
\end{remark} 

\begin{example}
    Examples of monoids are the following:
        \begin{enumerate}
            \item $\NN$: the natural numbers with addition $+$, where the special identity element is $0$: $0 + n = n + 0 = n$

            \item $(\ZZ / n, \times)$; the identity element is $1$
        \end{enumerate}
\end{example}

Now, we will start discussing groups, which we will stay on for half of the semester.

\subsection{Groups}
\begin{definition}
    A \vocab{group} $G$ is a monoid with one more operation, $i : G \to T$ written $i(x) = x^{-1}$ with the property that for any $x \in G$, $x \cdot x^{-1} = x^{-1} \cdot x$.
\end{definition}

In fact, it's enough to just require one-sided inverses $l : G \to G$ and $r : G \to G$ with $l(x)x = e$ and $xr(x) = e$. The reason is that
    \begin{align*}
        er(x) = (l(x)x)r(x) = l(x)(xr(x)) = l(x)e
    \end{align*}

We now present examples of groups with inversion:

\begin{example}
    The below are examples of groups with inversion:
    \begin{enumerate}
        \item $(\ZZ / n, +)$. The identity element is $0$. Inversion sends $x$ to $-x$. If you identify $\ZZ / n $ with $\{0, 1, \ldots, n-1\}$, addition and inversion have to be computed $\pmod n$. For example for $n = 4$, we have $3 + 3 = 6 \equiv 2 \pmod 4$; thus, the ``states" of $i$ are as follows: $0$ always returns to $2$, $1$ and $3$ communicate, and $2$ always remains at $2$.

        \item $((\ZZ / n)^x, x)$. The identity element is $1$. This group sits inside the monoid $(\ZZ(m, x))$. What equivalence classes lie in $(\ZZ / n)^x$? They are the equivalence classes of integers $x$ such that $\exists y, u$ with $yx = 1 + un$. This equation is equivalent to $yx-um = 1$, which is equivalent to saying $\text{gcd}(x, n) = 1$> If $n=p$ is prime, $(\ZZ / p)^x = \{1, 2, \ldots, p-1\}$.
    \end{enumerate}
\end{example}

You can compute inverses in $(\ZZ / n)^x$ by using Euclid's algorithm:

\begin{example}
    Consider $n = 13$, which is prime. What is the inverse of $6 \pmod {13}$? You need to solve for $y$ so that $by = 1 + 13u$. But $6 \cdot 2 = 12 = 13 - 1$, so $6 \cdot -2 = -12 = 1 - 13$. But $\pmod {13}$, $-2 = 13 - 2 = 11$, so $11$ is the multiplicative inverse of $6 \pmod {13}$.
\end{example}

\begin{definition}
    The \vocab{order} of a group $G$, denoted by $|G|$, is just its cardinality or number of elements.
\end{definition}

\begin{example}
    The below are examples of the order of a group:
    \begin{itemize}
        \item $|(\ZZ \ n, +)| = n$
        \item $|(\ZZ / n)^x, x)|$ is equal to the number of integers in $\{1, \ldots, n-1\}$ which have gcd $1$ with $n = \phi(n)$ (which is Euler's phi function). For $n$ prime, recall that $\phi(n) = n-1$.
    \end{itemize}
\end{example}

If $n = 8$, $(\ZZ / 8)^x = \{1, 3, 5, 7\}$ so $|(\ZZ / 8)^x| = 4$. IF $n = 12$, $(\ZZ / 12)^x = \{1, 5, 7, 11\}$, so $|(\ZZ / x)^x| = 4$. Some more examples of computing the order of a group are below:

\begin{example}
    The below are more examples of computing the order of a group:
    \begin{itemize}
        \item Consider the group $D_n$, where the operation is composition of symmetries. $|D_n| = 2n$.

        \item GL($2$, $\ZZ / p$) is the group of invertible $2 \times 2$ matrices with entries in $\ZZ / o$. The group operation is matrix multiplication; inversion is in the sense of matrices, e.g. 
        $\begin{pmatrix}
            a & b \\
            c & d
        \end{pmatrix} \in \text{GL}(2, \ZZ / p) \iff ad - bc != 0 \pmod p$. This is because the inverse of the matrix can be computed via 
        
        $\frac{1}{ad-bc} 
        \begin{pmatrix}
            d & -b \\
            -c & a
        \end{pmatrix}$ This is a noncommutative group. To find its order, note that $|\text{GL}(2, \ZZ / p)| = (p^2-1)(p^2-p) = p(p^2-1)(p-1)$. If $p = 2$, this is equal to $6$  
    \end{itemize}
\end{example}

\section{September 9, 2024}
Recall last class, where we started discussing groups and some examples of groups.

\begin{remark}
    In any group, inverses are unique. In fact, any one-sided inverse is automatically the two-sided inverse, i.e. if $xy=e$, then $yx = e$ comes for free. The reason for this is as follows: if $xy = e$, we can multiply by $y$ on the left, giving $(yx)y = y$, implying $yx$ must be the identity.
\end{remark}

\begin{remark}
    The group $\text{GL}(2, \ZZ / 2)$ has the same multiplication table as $D_3$. A multiplication table is a table that describes the structure of a (finite) group by arranging all the possible products of the group's elements (Wikipedia).
\end{remark}

\begin{definition}
    If $G$ is a group and $H \subseteq G$, $H$ is called a \vocab{subgroup} of $G$ if $H$ together with the group operations of $G$, is itself a group. This means $e \in H$ and $ab \in H$ for and $a, b, \in H$, and $a^{-1} \in H$ for all $a \in H$.
\end{definition}

\begin{remark}
    If $G$ is a group and $a \in G$, we can form $\left<a\right> = \{e, a, a^2, \ldots, a^{-1}, (a^{-1})^2 = a^{-2}, \ldots\}$. This is the smallest subgroup of $G$ containing $a$; $|\left<a\right>| = |a|$, the order of $a$. A subgroup of the form $<a>$, $a \in G$, is called \vocab{cyclic}.
\end{remark}

\begin{example}
    Another example of a subgroup is as follows: $G = \text{GL}(2, \ZZ / 3)$. This group has $(3^2-1)(3^2-3) = 48$ elements.
\end{example}

\section{September 11, 2024}

\end{document}