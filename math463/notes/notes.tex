\documentclass[12pt]{scrartcl}
\usepackage[sexy]{james}
\usepackage[noend]{algpseudocode}
\setlength {\marginparwidth}{2cm}
\usepackage{answers}
\usepackage{array}
\usepackage{tikz}
\newenvironment{allintypewriter}{\ttfamily}{\par}
\usepackage{listings}
\usepackage{xcolor}
\usetikzlibrary{arrows.meta}
\usepackage{color}
\usepackage{mathtools}
\newcommand{\U}{\mathcal{U}}
\newcommand{\E}{\mathbb{E}}
\usetikzlibrary{arrows}
\Newassociation{hint}{hintitem}{all-hints}
\renewcommand{\solutionextension}{out}
\renewenvironment{hintitem}[1]{\item[\bfseries #1.]}{}
\renewcommand{\O}{\mathcal{O}}
\declaretheorem[style=thmbluebox,name={Chinese Remainder Theorem}]{CRT}
\renewcommand{\theCRT}{\Alph{CRT}}
\setlength\parindent{0pt}
\usepackage{sansmath}
\usepackage{pgfplots}

\usetikzlibrary{automata}
\usetikzlibrary{positioning}  %                 ...positioning nodes
\usetikzlibrary{arrows}       %                 ...customizing arrows
\newcommand{\eqdef}{=\vcentcolon}
\newcommand{\tr}{{\rm tr\ }}
\newcommand{\im}{{\rm Im\ }}
\newcommand{\spann}{{\rm span\ }}
\newcommand{\Col}{{\rm Col\ }}
\newcommand{\Row}{{\rm Row\ }}
\newcommand{\dint}{\displaystyle\int}
\newcommand{\dt}{\ {\rm d }t}
\newcommand{\PP}{\mathbb{P}}
\newcommand{\C}{\mathcal{C}}
\newcommand{\horizontal}{\par\noindent\rule{\textwidth}{0.4pt}}
\usepackage[top=3cm,left=3cm,right=3cm,bottom=3cm]{geometry}
\newcommand{\mref}[3][red]{\hypersetup{linkcolor=#1}\cref{#2}{#3}\hypersetup{linkcolor=blue}}%<<<changed

\tikzset{node distance=4.5cm, % Minimum distance between two nodes. Change if necessary.
         every state/.style={ % Sets the properties for each state
           semithick,
           fill=cyan!40},
         initial text={},     % No label on start arrow
         double distance=4pt, % Adjust appearance of accept states
         every edge/.style={  % Sets the properties for each transition
         draw,
           ->,>=stealth',     % Makes edges directed with bold arrowheads
           auto,
           semithick}}


% Start of document.
\newcommand{\sep}{\hspace*{.5em}}

\pgfplotsset{compat=1.18}
\begin{document}
\title{MATH463: Complex Variables}
\author{James Zhang\thanks{Email: \mailto{jzhang72@terpmail.umd.edu}}}
\date{\today}

\definecolor{dkgreen}{rgb}{0,0.6,0}
\definecolor{gray}{rgb}{0.5,0.5,0.5}
\definecolor{mauve}{rgb}{0.58,0,0.82}

\lstset{frame=tb,
  language=Java,
  aboveskip=3mm,
  belowskip=3mm,
  showstringspaces=false,
  columns=flexible,
  basicstyle={\small\ttfamily},
  numbers=left,
  numberstyle=\tiny\color{gray},
  keywordstyle=\color{blue},
  commentstyle=\color{dkgreen},
  stringstyle=\color{mauve},
  breaklines=true,
  breakatwhitespace=true,
  tabsize=3
}


\maketitle
These are my notes for UMD's MATH463: Complex Variables. These notes are taken live in class 
(``live-\TeX``-ed). This course is taught by Professor Antoine Mellet.
\tableofcontents

\newpage

\section{Preliminaries}

\subsection{Definition of Complex Numbers}

\begin{definition}[Complex numbers]
    A \vocab{complex number} takes the form $z = x + iy$ where $x, y \in \RR$ and the $"x"$ is the 
    real part $= Re(z)$ and the $"y"$ is the imaginary part = $Im(z)$. We denote $\C$ to be the set of 
    complex numbers.
\end{definition}

\begin{note}
  Note that $i^2 = -1$. 
\end{note}

\begin{note}
  \[z_1 = z_2 \in \C \Longleftrightarrow \begin{cases}
    Re(z_1) = Re(z_2) \\ 
    Im(z_1) = Im(z_2)
  \end{cases}\]
\end{note}

\begin{definition}[Pure Imaginary Number]
  $Im(z) = 0 \implies z$ is real and $Re(z) = 0 \implies z$ is a \vocab{pure imaginary number}.
\end{definition}

\begin{note}
  $z = x + iy \in \C \Longleftrightarrow (x, y) \in \RR^2$
  and we draw this as the \vocab{complex plane}. We call the 
  x-axis the "real axis" and the y-axis the "imaginary axis."
\end{note}

\begin{example}
  Denote the set $S$ to be
  \[S = \{z \in \C \ | \ Re(z) = 2\} = \{z = 2 + iy \ | \ y \in \RR\}\]
  which graphically is a vertical line at $x = 2$. 
\end{example}

\subsection{Algebra}

\begin{definition}[Sum, Product of Complex Numbers]
  Let $z_1 = x_1 + iy_1, z_2 = x_2 + iy_2 \in \C$ then we can define the sum 
  \[z_1 \pm z_2 = (x_1 \pm x_2) + i(y_1 \pm y_2)\]
  \[z_1z_2 = (x_1 + iy_1)(x_2 + iy_2) = x_1x_2 + ix_1y_2 + ix_2y_1 - y_1y_2\]
  \[\implies z_1z_2 = (x_1x_2 - y_1y_2) + i(y_1x_2 + x_1y_2)\]
  \[z_1^2 = z_1z_1 = x_1^2 - y_1^2 + 2ix_1y_1\]
  We can continue to get more powers.
\end{definition}

\begin{definition}[Inverse]
  Given $z = x + iy$, we want to find $w = u + iv$ such that $zw = 1$, so $w = \frac{1}{z}$.
  \[zw = xu - yv + i(xv + yu) = 1 + i0 \implies \begin{cases}
    xu - yv = 1\\
    yu + xv = 0
  \end{cases}\]
  If you know $x$ and $y$, then this is just a linear system with $2$ unknowns. We can rewrite this as 
  \[\begin{bmatrix}
    x & -y \\ y & x
  \end{bmatrix} \begin{bmatrix}
    u \\ v
  \end{bmatrix} = \begin{bmatrix}
    1 \\ 0
  \end{bmatrix}\]
  Therefore, this system has a unique soltion if and only if the determinant of this matrix $= x^2 + y^2 \neq 0 \implies x \neq 0 \ \& y \ \neq 0$. 

  Thus, $z$ has a unique inverse if and only if $z \neq 0 + i0$ and 
  \begin{align*}
    u = \frac{x}{x+2 + y^2} & v = -\frac{y}{x^2 + y^2}
  \end{align*}
  So, 
  \[\frac{1}{z} = \frac{x}{x^2 + y^2} - i\frac{y}{x^2 + y^2}\]
\end{definition}

\begin{definition}[Fraction]
  Following inverses, fractions naturally follow. For example, $\frac{z_1}{z_2} = z_1 \cdot \frac{1}{z_2}$
\end{definition}

\begin{definition}[Projection] $(z_1z_2)^{-1} = z_1^{-1}z_2^{-1}$
\end{definition}

\subsection{Vectors and Modulus}

\begin{definition}[Vector Representations]
  For example, if $z_1 = 3 - 2i$ then in vector representation this appears as the vector 
  $\begin{bmatrix}
    3 & -2
  \end{bmatrix}' \in \RR^2$
\end{definition}

\begin{definition}[Modulus]
  The modulus of $z = x + iy$ is the norm of the vector $\begin{bmatrix}
    x & y
  \end{bmatrix}' \implies |z| = \sqrt{x^2 + y^2} \in \RR, |z| \geq 0$
\end{definition}

\begin{note}
  $|z| = 0$ if and only if $z = 0 + i0$
\end{note}

\begin{note}
  If $y = 0 \implies z = x \implies |x| = \sqrt{x^2} = |x|$. Thus, when $y=0$, the modulus of $z$ is the absolute value of $x$
\end{note}

\begin{example}
  $|3-2i| = \sqrt{3^2 + 2^2} = \sqrt{13}$
\end{example}

\begin{example}
  $|i| = \sqrt{0^2 + i^2} = 1$
\end{example}

\begin{remark}
  Observe that 
  \[|z| = \sqrt{x^2 + y^2} \geq \sqrt{x^2} = |Re(z)|\]
  \[|z| = \sqrt{x^2 + y^2} \geq \sqrt{y^2} = |Im(z)|\]
\end{remark}

\begin{definition}[Distance]
  The distance between $z_1 = x_1 + iy_1$ and $z_2 = x_2 + iy_2$ is 
  \[d = \sqrt{(x_2 - x_1)^2 + (y_2 - y_1)^2} = |(x_2 - x_1) + i(y_2 - y_1)| = |z_2 - z_1|\]
\end{definition}

\begin{definition}[Circles]
  A cricle centered at $z_0 = x_0 + iy_0$ with radius $R$ can be expressed as the set 
  \[C = \{z \in \C \ | \ |z - z_0| = R\} = \{z \in \C \ | \ (x-x_0)^2 + (y-y_0)^2 = R^2\}\]
  and so we recover the classical definition of a circle.
\end{definition}

\begin{example}
  Define the set $S_1 = \{z \in \C \ | \ |z| = 2\}$ which is a circle of radius $2$ centered at 
  $z_0 = 0$. Now define $S_2 = \{z \in \C \ | \ |z - (-2+ 3i)| = 2\}$ which is a circle of radius $2$ 
  centered at $-2 + 3i$.
\end{example}

\begin{definition}[Disk]
  Consider $D = \{z \in \C \ | \ |z-1| \leq 3\}$ which is a disk centered at $z_0 = 1, R=3$.
\end{definition}

\begin{theorem}[Triangle Inequality]
  Consider two complex numbers $z_1, z_2$ then 
  \[|z_1 + z_2| \leq |z_1| + |z_2| \ \forall \ z_1, z_2 \in \C\]

  \begin{proof}
    Proof ommitted.
  \end{proof}
\end{theorem}

\begin{corollary}[Consequences of Triangle Inequality]
  \hfill
  \begin{itemize}
    \item $|z_1 + z_2 + \cdots + z_n| \leq |z_1| + |z_2| + \cdots + |z_n|$
    \item $|z_1 + z_2| \geq \left| |z_1| - |z_2|\right|$
  \end{itemize}

  \begin{proof}
    Note that $z_1 = z_1 + z_2 +(-z_2)$ and so 
    \[|z_1| = |(z_1 + z_2) + (-z_2) \leq |z_1 + z_2| + |-z_2| = |z_1 + z_2| + |z_2|\]a
    and so we have showed that $|z_1| - |z_2| \leq |z_1 + z_2|$. The absolute value is necessary 
    in case $|z_2| > |z_1|$, in which case we can perform a similar proof by switching the order of $z_1$ and $z_2$.
  \end{proof}
\end{corollary}

\begin{example}
  Consider the unit circle $\{z \in \C \ | \ |z| = 1\}$ and the point $z_0 = 2$. The smallest this distance can be is $1$, 
  and the largest is $3$.
  \[|z - z_0| \leq |z| + |-z_0| = 1 + 2 = 3 \text{ by Triangle Inequality}\]
  \[|z-z_0| \geq \left| |z| - |z_0| \right| = |1 - 2| = 1 \text{ by Corollary}\]
\end{example}

\subsection{Conjugate of a Complex Number}

\begin{definition}[Conjugate of a Complex Number]
    If $z = x + iy$ then the \vocab{conjugate of $z$} is denoted $\bar{z} = x - iy$.
\end{definition}

\begin{note}[Properties of Conjugates]
  
  \hfill

  \begin{itemize}
    \item $\overline{z_1 + z_2} = \bar{z_1} + \bar{z_2}$
    \item $\overline{z_1z_2} = \bar{z_1} * \bar{z_2}$
    \item $\bar{(\frac{1}{z_1})} = \frac{1}{\bar{z_1}}$
    \item $\overline{\bar{z}} = z$
  \end{itemize}

\end{note}

\begin{remark}
  
  \hfill

  \begin{itemize}
    \item $z + \overline{z} = x + iy + x - iy = 2x$
    \item $z - \overline{z} = 2iy$
    \item $Im(z) = \frac{1}{2i}(z - \bar{z})$
    \item $z * \bar{z} = (x + iy)(x - iy) = x^2 + y^2 = |z|^2 \geq 0$
  \end{itemize}
\end{remark}

\begin{note}[Applications of the Product of a Complex Number and its Conjugate]

  \hfill

  \begin{enumerate}
    \item $\frac{z_1}{z_2} = \frac{z_1 \overline{z_2}}{z_2 \overline{z_2}} = \frac{z_1\overline{z_2}}{|z_2|^2}$
    \item $|z_1 z_2|^2 = z_1z_2 (\overline{z_1z_2}) = z_1z_2\overline{z_1}\overline{z_2} = (z_1\overline{z_1})(z_2 \overline{z_2}) = |z_1|^2 |z_2|^2 \implies |z_1z_2| = |z_1||z_2|$
    \item $|\frac{z_1}{z_2}| = \frac{|z_1|}{|z_2|}$
    \item $|z^2| = |z|^2$ and $|z^n| = |z|^n \ \forall \ n = \pm 1, \pm 2, \cdots$
  \end{enumerate}
\end{note}

\subsection{Exponential Form of Complex Numbers}

\begin{definition}
  Complex numbers $z \neq 0$ also have polar coordinates $(r, \theta)$ and the results 
  $x = r\cos\theta \implies \cos\theta = \frac{x}{r}$, $y = r\sin\theta \implies \sin\theta = \frac{y}{r}$, $r = \sqrt{x^2 + y^2}$ still holds. 
  Note that $r \RR^+$ and $\theta \in \RR$ and is the angle in radians.
\end{definition}

\begin{note}
  Note that $r = \sqrt{x^2 + y^2} = |z|$ is the modulus of $z$. 
\end{note}

\begin{definition}[Argument of a Complex Number]
  $\theta$ is called the \vocab{argument of $z$} and is denoted as $\arg(z)$, which is a multivalued function because $\theta$ is not uniquely defined. You can think of it as 
  the $z$ here being a set. Note that if 
  $\theta$ is an argument, then $\theta + 2k\pi, k = \pm 1, \pm 2, \cdots$ is also still an argument.
\end{definition}

\begin{definition}[Principal Argument]
  $\text{Arg}(z)$ is the \vocab{principal argument}, and it is the only argument of $z$ in the interval $[-\pi, \pi]$.
\end{definition}

\begin{example}
  $\text{Arg}(1) = 0$, $\text{Arg}(-1) = \pi$, $\text{Arg}(i) = \frac{\pi}{2}$, $\text{Arg}(-5i) = -\frac{\pi}{2}$
\end{example}

\begin{example}
  $\text{Arg}(\sqrt{3} + i) = \frac{\pi}{6}$
  \begin{proof}[Solution]
    $z = \sqrt{3} + i$. Note that $r = |z| = \sqrt{\sqrt{3}^2 + 1^2} = 2$. Since $\cos\theta = \frac{x}{r} \implies \cos\theta = \frac{\sqrt{3}}{2} = \frac{\pi}{6}$.
  \end{proof}
\end{example}

\begin{definition}[Euler's Formula]
  \[e^{i\theta} = \cos\theta + i\sin\theta\]
  If $z$ has modulus $r$, argument $\theta$ then 
  \begin{align*}
    z = Re(z) + iIm(z) = r\cos\theta + ir\sin\theta = r(\cos\theta + i\sin\theta) = re^{i\theta}
  \end{align*}
\end{definition}

\begin{example} Some cases of the above formula

  \begin{itemize}
    \item $\sqrt{3} + i = 2e^{i\pi / 6}$
    \item $1 + i = \sqrt{e^{i\pi / 4}}$
    \item $3 = 3e^{i0}$
    \item $-5 = 5e^{i\pi} \implies -1 = e^{i\pi}$
    \item $i = e^{i \pi / 2}$
  \end{itemize}
\end{example}

\begin{example}
  Given the form of a complex number in this form $z = re^{i\theta}$ we can rewrite it using Euler's Formula.
  \[z = e^{i\pi / 3} = 3\cos\frac{\pi}{3} + i3\sin\frac{\pi}{3} = \frac{3}{2} + i\frac{3 \cdot \sqrt{3}}{2}\]
\end{example}

\begin{note}
  We can define circles using this form $\{re^{i\theta} \ | \ \theta \in \RR\} = \{r^{i\theta} \ | \ \theta \in [0, 2\pi]\}$ and this is just a circle centered at the 
  origin with radius $r$. 
\end{note}

\begin{note}
  For circles not centered at the origin, say $z_0$ with radius $R$, recall that conventionally
  \begin{align*}
    |z-z_0| = R \implies z-z_0 = Re^{i\theta} \implies z = z_0 + Re^{i\theta}\\ 
    c = \{z_0 + Re^{i\theta} \ | \ \theta \in [-\pi, \pi]\}
  \end{align*}
\end{note}

\begin{note}[Products in Exponential Form]
  \begin{align*}
    e^{i\theta_i} \cdot e^{i\theta_2} = (cos\theta_1 + i\sin\theta_1)(\cos\theta_2 + i\sin\theta_2)\\
    = (\cos\theta_1 \cos\theta_2 - \sin\theta_1 \sin\theta_2) + i(\cos\theta_1 \sin\theta_2 + \cos\theta_2\sin\theta_1)\\
     = \cos(\theta_1 + \theta_2) + i\sin(\theta_1 + \theta_2) = e^{i(\theta_1 + \theta_2)}
  \end{align*}
\end{note}

\begin{note}
  If $z_1 = r_1e^{i\theta_1}, z_2 = r_2e^{\theta_2}$ not zero, then 
  \[z_1z_2 = r_1r_2e^{i(\theta_1 + \theta_2)}\]
  This shows us the following 
  \begin{align*}
    |z_1z_2| = |z_1||z_2| && \arg(z_1z_2) = \arg(z_1) + \arg(z_2)
  \end{align*}
  Note that this is lowercase $\arg$, and we present the following counterexample. Let $z_1 = -1, z_2 = i, z_1z_2 = -i$
  \begin{align*}
    \text{Arg}(z_1) = \pi && \text{Arg}(z_2) = \frac{\pi}{2} && \text{Arg}(z_1z_2) = -\frac{\pi}{2}
  \end{align*}
\end{note}

\begin{example}
  $(1 + i)^4 = (\sqrt{2}e^{i\pi/4}) = (\sqrt{2})^4 e^{i4\pi / 4} = 4e^{i\pi} = -4$
\end{example}

\begin{theorem}[De Moivre's Formula]
  \[(\cos\theta + i\sin\theta)^n = \cos(n\theta) + i\sin(n\theta)\]
\end{theorem}

\begin{definition}[Inverse of $z$ in Exponential Form]
  Let $z = re^{i\theta}$ and so $z^{-1} = r^{-1}e^{-i\theta}$. We can verify this because 
  \[(re^{i\theta})(r^{-1}e^{-i\theta}) = (r \cdot r^{-1})e^{i(\theta - \theta)} = 1\]
\end{definition}

\begin{example}
  $(\sqrt{3} + i)^{-1} = \frac{1}{\sqrt{3} + i} = \frac{1}{2e^{i\pi / 6}} = \frac{1}{2}e^{-i\pi / 6}$
\end{example}

\begin{definition}[Fractions]
  $\frac{z_1}{z_2} = z_1 \cdot \frac{1}{z_2} \implies \frac{r_1e^{i\theta_1}}{r_2e^{i\theta_2}} = \frac{r_1}{r_2}e^{i(\theta_1 - \theta_2)}$
  This tells us that 
  \begin{align*}
    |\frac{z_1}{z_2}| = \frac{|z_1|}{|z_2|} && \arg(\frac{z_1}{z_2}) = \arg(z_1) - \arg(z_2)
  \end{align*}
\end{definition}

\begin{example}
  Find the $\text{Arg}(\frac{-5}{\sqrt{3} + i}) = \frac{5\pi}{6}$

  \begin{proof}[Solution]
    Recall that 
    \[\arg(\frac{-5}{\sqrt{3} + i}) = \arg(-5) - \arg(\sqrt{3} + i) = \pi - \frac{\pi}{6} + 2k\pi = \frac{5\pi}{6} + 2k\pi\]
    and so the Principal Argument is $\frac{5\pi}{6}$.
  \end{proof}
\end{example}

\begin{note}
  Note that for all $n \in \NN$, 
  \begin{align*}
    z^n = r^n e^{in\theta} && z^{-n} = r^{-n}e^{-in\theta}
  \end{align*}
  and so therefore 
  \begin{align*}
    |z^n| = |z|^n && \arg(z^n) = n\arg(z) + 2k\pi \text{ for } n = \pm 1, \pm 2, \cdots
  \end{align*}
\end{note}

\begin{definition}[Conjugate in Exponential Form]
  Recall that $|z| = |\overline{z}|$ and $\arg(\overline{z}) = -\arg(\overline{z})$ and so therefore if $z = re^{i\theta}$
  \[\overline{z} = re^{-i\theta}\]
  and $z^{-1} = r^{-1}e^{-i\theta}$
\end{definition}

\end{document}

